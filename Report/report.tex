% This is samplepaper.tex, a sample chapter demonstrating the
% LLNCS macro package for Springer Computer Science proceedings;
% Version 2.20 of 2017/10/04
%
\documentclass[runningheads]{llncs}
%
\usepackage{graphicx}
% Used for displaying a sample figure. If possible, figure files should
% be included in EPS format.
\usepackage{hyperref}
% If you use the hyperref package, please uncomment the following line
% to display URLs in blue roman font according to Springer's eBook style:
\renewcommand\UrlFont{\color{blue}\rmfamily}


\begin{document}
%
\title{Analysis of and Mitigation Strategies for \\Real World ICS Security Incidents}
%
%\titlerunning{Abbreviated paper title}
% If the paper title is too long for the running head, you can set
% an abbreviated paper title here
%
\author{Nico Fechtner}
%
\authorrunning{N. Fechtner}
% First names are abbreviated in the running head.
% If there are more than two authors, 'et al.' is used.
%
\institute{Technical University of Munich \& \\Fraunhofer Institute for Applied and Integrated Security\\
\href{mailto:nico.fechtner@tum.de}{nico.fechtner@tum.de}}
%
\maketitle              % typeset the header of the contribution
%
\begin{abstract}
% The abstract should briefly summarize the contents of the paper in
% 150--200 words.
% The goal is to communicate: Background/motivation/context, Aim/objective(s)/problem statement, Approach/method(s)/procedure(s), Results, Conclusion(s)/implications.
As more and more Industrial Control Systems (ICS) are getting connected to the internet and IT networks---intentionally or by mistake---the attack surface of these systems increases dramatically.
Due to this, the number of real world ICS security incidents rises, too, which is why it is crucial to develop efficient detection and mitigation strategies.
A solid baseline in this process is analyzing and learning from past incidents.
This is crucial to avoid common mistakes and to effectively prevent future incidents.
To aid this process, this paper proposes a systematic comparative overview of historic ICS security incidents focussing on various parameters including targets, threat actors, attack techniques, goals and impacts, dwell times, operator reactions and detection and mitigation strategies.
Several key findings are resulting from this overview and an in-depth analysis of the Triton incident.
First, x.
Second, y.
Third, z.

\keywords{ICS Security \and Security Incident \and Triton.}
\end{abstract}
%
%
%

\section{Introduction}
% The introduction supplies sufficient background information for the reader to understand and evaluate the work you did. Assume the reader is a generic computer science student, knowledgeable and acquainted with all basic computer science concepts. The introduction should also include a description of the problem you want to solve, your research questions, and an outline for the remainder of your paper. The goal is to: Indicate the field of the work, why this field is important, and what has already been done (with proper citations), Outline the purpose of your paper and announce your research questions, Avoid: repeating the abstract; providing unnecessary background information.
While Information Technology (IT) refers to software and hardware that generate data for enterprise use, Operational Technology (OT) describes software and hardware able to detect or cause a physical event in an industrial environment.
% ICS
Probably the most important subcategory of OT, at least revenue-wise, are Industrial Control Systems (ICS).
They are used in a wide variety of industries such as food and agriculture, energy, water, transportation, chemical, nuclear power, pharmaceutical and discrete manufacturing \cite{stouffer.2011}.\\
% OT Security <-> IT Security: very recent trend, first incidents
Initially, ICS were not connected to the internet and strictly separated from IT networks.
Due to this isolation, it was hard if not impossible for adversaries to remotely attack ICS which is why until recently security was not a big concern to companies running ICS.
Instead, they traditionally focus on safety, continuity and efficiency of their systems.
However, the attack surface of many ICS changed within the last two decades since they got connected to traditional IT networks and the internet---sometimes intentionally, sometimes by mistake.
Inevitably, this led to an ongoing series of ICS security incidents.\\
% Problem: No overview; Contribution: Overview
While more and more of those incidents are reported, it is business-critical to develop suitable detection and mitigation strategies.
A solid baseline in this process is analyzing and learning from past incidents.
This is crucial to avoid common mistakes and to effectively prevent future incidents.
To aid this process, this paper proposes a systematic comparative overview of historic ICS security incidents focussing on various parameters including targets, threat actors, attack techniques, goals and impacts, dwell times, operator reactions and detection and mitigation strategies.
% Benefits of an overview: Helps to identify patterns and to prevent future incidents (?)
The overview aims at identifying common attack patterns that could be used to prevent future attacks.
To the best of the author's knowledge, such an overview has not yet been published.\\
% Triton
In addition, the Triton incident of 2017 will be analyzed in detail to showcase how attackers perform sophisticated ICS attacks and which detection and mitigation strategies can be derived from their methodologies.
The incident was chosen due to its potential life-threatening impact and the novel attack approach targeting Safety Instrumented Systems (SIS).\\
% Outline
The remainder of the paper is structured as follows.
Section 2 provides an overview of related work in the area of ICS incidents which is used as the basis of the comparative overview that is provided in Section 3.
Exemplary the Triton incident is analyzed in detail in Section 4.
Section 5 concludes the paper by emphasizing the need to learn from past mistakes.
\section{Related Work}
% In this section, you provide an overview of papers written by other scientists that have covered a problem similar to yours. You can find related work in web search engines and scientific literature databases. Some of the most prominent ones in the field of computer science are: Google Scholar, Scopus, ACM Digital Library, IEEE Xplore, Lecture Notes in Computer Science.
There are four main types of publicly available sources that provide information on ICS security incidents.
% Enumerations: RISI Database, ICS-CERT Alerts from the US Government,
First, there are incident enumerations like the Risi Database\footnote{https://www.risidata.com/Database} and the ICS-CERT Alerts\footnote{https://www.us-cert.gov/ics/alerts}.
On the one hand, the fact that those repositories aim to aggregate all observed ICS incidents from around the world makes them useful for getting a high-level overview of the current threat landscape.
On the other hand, however, they only provide basic information about the incidents and do not analyze them in detail.
% Reports from Companies in the field: Dragos Threat Report, CyberX Global IoT/ICS Risk Report,...
Second, ICS security companies like Dragos \cite{dragos.19} and CyberX \cite{cyberx.19} publish yearly ICS threat reports discussing relevant incidents.
Those reports are neither complete with regards to the incidents they cover nor do they provide in-depth analyses of the covered attacks.
However, they do a good job of highlighting trends in the threat landscape throughout the years.
Third, there are dedicated scientific papers focussing on single ICS incidents like the attacks targeting the Ukraine power grid \cite{eisac.16} or the Triton malware \cite{pinto.18}. Those papers usually cover single incidents in-depth and help in understanding how exactly the adversaries operated.
Fourth, there are numerous blog posts, press releases and conference talks covering ICS incidents.
In addition, there is already a paper proposing an overview of historical ICS security incidents \cite{hemsley.18}.
However, the paper is slightly dated and, more importantly, does not compare the different incidents systematically.
All of the above-mentioned sources are taken into account to achieve exactly this in the upcoming section.

% Other papers covering the big incidents: Ukraine, Triton,...
% IT incidents overview
\section{Comparative Overview of ICS Security Incidents}
From the resources and literature stated in Section 2, a total of 29 ICS security incidents were extracted.
Those form the basis of the following analysis and can be found in table [appendix table].
Note that this table is inherently incomplete, but rather tries to focus on the most prevalent attacks.
In the subsections below, a comparative overview of these incidents will be given.
Each subsection tries to compare the incidents with regards to specific attributes, e.g. the involved threat actors or the utilized attack techniques.
Most subsections contain both quantitative as well as qualitative analyses.
Please keep in mind, though, that the quantitative analyses are solely intended to highlight important trends and that the according absolute values might contains inaccuracies.
This is due to the fact that often specific information about incidents is unknown.
For example, consider the dwell time which describes how long an adversary is in a system before the actual attack payload is executed.
Sometimes it is possible to give an accurate estimation for this number thanks to digital forensics after an incident.
In general, however, this is a non-trivial task and even if performed successfully, it may still lead to incorrect conclusions about the dwell time and other incident attributes.

\subsection{Targets}
% Targeted vs. Untargeted
When analyzing the entities affected by ICS security incidents there are two fundamentally different kinds of attacks to be considered.
On the one hand, there are targeted attacks utilized to breach specific entities.
Making up roughly 86\% of the analyzed incidents, targeted attacks seem to be present a very diverse threat landscape to ICS and will therefore be analyzed in detail below.
On the other hand, there are untargeted attacks.
While only a few of the analyzed incidents fall in this category, they still pose a significant threat to ICS.
Interesting to see is that the malware used for untargeted attacks usually is not tailored specifically for ICS environments, but rather targets common operating systems like Microsoft Windows in general.
Many instances of untargeted attacks fall in the category of ransomware.
For example, the popular WannaCry ransomware spread not only to desktop computers but also infected a series of ICS workstations which lead to outages due to encrypted hard drives (?) [citation].\\
% Untargeted
Since untargeted attacks are not aimed at specific entities there is no pattern observable in terms of the geographical location or the economic sector of the affected parties.
Targeted attacks, however, can be analyzed for such patterns.\\
% Country
When it comes to the geographical location of ICS attack victims, the most targeted areas seem to be the Middle East and Europe. The Middle East is being targeted by about 56\% of the analyzed attacks. Especially companies located in Saudi Arabia often fall victim to attacks. Probably the most well-known incident taking place there was Triton which targeted a Saudi Arabian petrochemical plant and will be covered in depth in Section 4.
Companies located in Europe are targeted by about 52\% of the analyzed attacks. The country affected the most until now is Ukraine falling victim to multiple attacks targeting its power grid.
In about 24\% of the analyzed incidents, the US were amongst the victims.
Interesting is that while being home to important global threat actors, neither Russia nor China reports a lot of ICS attacks.
However, this does not necessarily mean that no attacks against these countries are taking place, but it could also be the case that incidents are just not as liberally published as by other countries.
Especially China is known for withholding most of the cyber attacks taking place in their country [citation].\\
% Sector (+ private/public)
When it comes to the economic sector falling victim to targeted ICS attacks, the most affected one is the energy sector being targeted in about 52\% of the analyzed targeted attacks.
Most and foremost, those operations include attacks against power grids like the attacks taking place in 2014 and 2015 in Ukraine.
Other examples of targeted entities within the energy sector include oil and gas pipelines and refineries.
The remaining victims of ICS attacks are spread across a wide variety of other economic sectors including manufacturing, water, petrochemical, governmental organizations and transportation.

\subsection{Threat Actors}
\subsection{Attack Techniques}
\subsection{Attack Goals and Realized Impacts}
\subsection{Dwell Times and Reactions}
\subsection{Detection and Mitigation Strategies}
\section{Detailed Analysis of the Triton Attack}
\subsection{The Petro Rabigh Oil Refinery}
\subsection{The Attackers' Approach}
\subsection{Impact}
\subsection{Attribution}
\subsection{Detection and Mitigation Opportunities}
\section{Conclusion}
% The conclusion summarizes the research and discusses its significance. You should also point out future research directions. The goal is to: Provide a very brief summary of the results, Provide a future perspective on the work, Avoid: repeating the abstract; repeating background information from the introduction; introducing new arguments; repeating the arguments made in the main body; failing to ad- dress all of the research questions set out in the introduction.
% Summary
% Learn from past mistakes
% Future Work


%
% ---- Bibliography ----
%
% BibTeX users should specify bibliography style 'splncs04'.
% References will then be sorted and formatted in the correct style.
%
\bibliographystyle{splncs04}
\bibliography{bibliography}
\end{document}
