% This is samplepaper.tex, a sample chapter demonstrating the
% LLNCS macro package for Springer Computer Science proceedings;
% Version 2.20 of 2017/10/04
%
\documentclass[runningheads]{llncs}
%
\usepackage{graphicx}
% Used for displaying a sample figure. If possible, figure files should
% be included in EPS format.
\usepackage{hyperref}
% If you use the hyperref package, please uncomment the following line
% to display URLs in blue roman font according to Springer's eBook style:
\renewcommand\UrlFont{\color{blue}\rmfamily}


\begin{document}
%
\title{Analysis of and Mitigation Strategies for \\Real World ICS Security Incidents}
%
%\titlerunning{Abbreviated paper title}
% If the paper title is too long for the running head, you can set
% an abbreviated paper title here
%
\author{Nico Fechtner}
%
\authorrunning{N. Fechtner}
% First names are abbreviated in the running head.
% If there are more than two authors, 'et al.' is used.
%
\institute{Technical University of Munich \& \\Fraunhofer Institute for Applied and Integrated Security\\
\href{mailto:nico.fechtner@tum.de}{nico.fechtner@tum.de}}
%
\maketitle              % typeset the header of the contribution
%
\begin{abstract}
% The abstract should briefly summarize the contents of the paper in
% 150--200 words.
% The goal is to communicate: Background/motivation/context, Aim/objective(s)/problem statement, Approach/method(s)/procedure(s), Results, Conclusion(s)/implications.
As more and more Industrial Control Systems (ICS) are getting connected to the internet and IT networks---intentionally or by mistake---the attack surface of these systems increases dramatically.
Due to this, the number of real world ICS security incidents rises, too, which is why it is crucial to develop efficient detection and mitigation strategies.
A solid baseline in this process is analyzing and learning from past incidents.
This is crucial to avoid common mistakes and to effectively prevent future incidents.
To aid this process, this paper proposes a systematic comparative overview of historic ICS security incidents focussing on various parameters including targets, threat actors, attack techniques, goals and impacts, dwell times, operator reactions and detection and mitigation strategies.
Several key findings are resulting from this overview and an in-depth analysis of the Triton incident.
First, x.
Second, y.
Third, z.

\keywords{ICS Security \and Security Incident \and Triton.}
\end{abstract}
%
%
%

\section{Introduction}
% The introduction supplies sufficient background information for the reader to understand and evaluate the work you did. Assume the reader is a generic computer science student, knowledgeable and acquainted with all basic computer science concepts. The introduction should also include a description of the problem you want to solve, your research questions, and an outline for the remainder of your paper. The goal is to: Indicate the field of the work, why this field is important, and what has already been done (with proper citations), Outline the purpose of your paper and announce your research questions, Avoid: repeating the abstract; providing unnecessary background information.
While Information Technology (IT) refers to software and hardware that generate data for enterprise use, Operational Technology (OT) describes software and hardware able to detect or cause a physical event in an industrial environment.
% ICS
Probably the most important subcategory of OT, at least revenue-wise, are Industrial Control Systems (ICS).
They are used in a wide variety of industries such as food and agriculture, energy, water, transportation, chemical, nuclear power, pharmaceutical and discrete manufacturing \cite{stouffer.2011}.\\
% OT Security <-> IT Security: very recent trend, first incidents
Initially, ICS were not connected to the internet and strictly separated from IT networks.
Due to this isolation, it was hard for adversaries to remotely attack ICS which is why until recently security was not a big concern to companies running ICS.
Instead, they traditionally focus on safety, continuity and efficiency of their systems.
However, the attack surface of many ICS changed within the last two decades since they got connected to traditional IT networks and the internet---sometimes intentionally, sometimes by mistake.
Inevitably, this led to an ongoing series of ICS security incidents.\\
% Problem: No overview; Contribution: Overview
While more and more of those incidents are reported, it is business-critical to develop suitable detection and mitigation strategies.
A solid baseline in this process is analyzing and learning from past incidents.
This is crucial to avoid common mistakes and to effectively prevent future incidents.
To aid this process, this paper proposes a systematic comparative overview of historic ICS security incidents focussing on various parameters including targets, threat actors, attack techniques, goals and impacts, dwell times and reactions and detection and mitigation strategies.
% Benefits of an overview: Helps to identify patterns and to prevent future incidents (?)
The overview aims at identifying common attack patterns that could be used to prevent future attacks.
To the best of the author's knowledge, such an overview was not yet published.\\
% Triton
In addition, the Triton incident of 2017 will be analyzed in detail to showcase how attackers perform sophisticated ICS attacks and which detection and mitigation strategies can be derived from this process.
The incident was chosen due to its potential life-threatening impact and the novel attack approach targeting Safety Instrumented Systems (SIS).\\
% Outline
The remainder of the paper is structured as follows.
Section 2 provides an overview of related work in the area of ICS incidents which is used as the basis of the comparative overview that is provided in Section 3.
Exemplary the Triton incident is analyzed in detail in Section 4.
Section 5 concludes the paper by emphasizing the need to learn from past mistakes.
\section{Related Work}
% In this section, you provide an overview of papers written by other scientists that have covered a problem similar to yours. You can find related work in web search engines and scientific literature databases. Some of the most prominent ones in the field of computer science are: Google Scholar, Scopus, ACM Digital Library, IEEE Xplore, Lecture Notes in Computer Science.
% Enumerations: RISI Database, ICS-CERT / CISA (https://www.us-cert.gov/ncas/alerts/2020) Alerts from the US Government,
% Reports from Companies in the field: Dragons Threat Report, CyberX Global IoT/ICS Risk Report,...
% Other papers covering the big incidents: Ukraine, Triton,...
\section{Comparative Overview of ICS Security Incidents}
\subsection{Targets}
% Country, Target, Sector, System Type
\subsection{Threat Actors}
\subsection{Attack Techniques}
\subsection{Attack Goals and Realized Impacts}
\subsection{Dwell Times and Reactions}
\subsection{Detection and Mitigation Strategies}
\section{Detailed Analysis of the Triton Attack}
\subsection{The Petro Rabigh Oil Refinery}
\subsection{The Attackers' Approach}
\subsection{Impact}
\subsection{Attribution}
\subsection{Detection and Mitigation Opportunities}
\section{Conclusion}
% The conclusion summarizes the research and discusses its significance. You should also point out future research directions. The goal is to: Provide a very brief summary of the results, Provide a future perspective on the work, Avoid: repeating the abstract; repeating background information from the introduction; introducing new arguments; repeating the arguments made in the main body; failing to ad- dress all of the research questions set out in the introduction.
% Summary
% Learn from past mistakes
% Future Work


%
% ---- Bibliography ----
%
% BibTeX users should specify bibliography style 'splncs04'.
% References will then be sorted and formatted in the correct style.
%
\bibliographystyle{splncs04}
\bibliography{bibliography}
\end{document}
