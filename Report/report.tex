% This is samplepaper.tex, a sample chapter demonstrating the
% LLNCS macro package for Springer Computer Science proceedings;
% Version 2.20 of 2017/10/04
%
\documentclass[runningheads]{llncs}
%
\usepackage{graphicx}
% Used for displaying a sample figure. If possible, figure files should
% be included in EPS format.
%
% If you use the hyperref package, please uncomment the following line
% to display URLs in blue roman font according to Springer's eBook style:
% \renewcommand\UrlFont{\color{blue}\rmfamily}

\begin{document}
%
\title{Analysis of and Mitigation Strategies for Real World OT Security Incidents}
%
%\titlerunning{Abbreviated paper title}
% If the paper title is too long for the running head, you can set
% an abbreviated paper title here
%
\author{Nico Fechtner}
%
\authorrunning{N. Fechtner}
% First names are abbreviated in the running head.
% If there are more than two authors, 'et al.' is used.
%
\institute{Technical University of Munich, Arcisstra{\ss}e 21, 80333 Munich, Germany
\email{nico.fechtner@tum.de}\\
\url{https://www.tum.de/en/}}
%
\maketitle              % typeset the header of the contribution
%
\begin{abstract}
The abstract should briefly summarize the contents of the paper in
150--200 words.
The goal is to communicate: Background/motivation/context, Aim/objective(s)/problem statement, Approach/method(s)/procedure(s), Results, Conclusion(s)/implications.

\keywords{First keyword  \and Second keyword \and Another keyword.}
\end{abstract}
%
%
%

\section{Introduction}
The introduction supplies sufficient background information for the reader to understand and eval- uate the work you did. Assume the reader is a generic computer science student, knowledgeable and acquainted with all basic computer science concepts. The introduction should also include a description of the problem you want to solve, your research questions, and an outline for the remainder of your paper. The goal is to: Indicate the field of the work, why this field is important, and what has already been done (with proper citations), Outline the purpose of your paper and announce your research questions, Avoid: repeating the abstract; providing unnecessary background information.
\section{Related Work}
In this section, you provide an overview of papers written by other scientists that have covered a problem similar to yours. You can find related work in web search engines and scientific literature databases. Some of the most prominent ones in the field of computer science are: Google Scholar, Scopus, ACM Digital Library, IEEE Xplore, Lecture Notes in Computer Science.
\section{Real-world Attacks on OT Environments}
\subsection{Comparative Overview}
\subsubsection{Methodology}
\subsubsection{Results}
\subsubsection{Quantitative Discussion}
\subsubsection{Qualitative Discussion}
\subsection{Detailed Analysis of the Triton Attack}
\subsubsection{Stage 1}
\subsubsection{Stage 2}
\subsubsection{Stage 3}
\section{Conclusion}
The conclusion summarizes the research and discusses its significance. You should also point out future research directions. The goal is to: Provide a very brief summary of the results, Provide a future perspective on the work, Avoid: repeating the abstract; repeating background information from the introduction; introducing new arguments; repeating the arguments made in the main body; failing to ad- dress all of the research questions set out in the introduction.
%
% ---- Bibliography ----
%
% BibTeX users should specify bibliography style 'splncs04'.
% References will then be sorted and formatted in the correct style.
%
\bibliographystyle{splncs04}
\bibliography{bibliography.bib}
\end{document}
